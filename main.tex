\documentclass{amsart}

\usepackage{graphicx}
\renewcommand{\vec}{\textbf}

\theoremstyle{definition}
\newtheorem{definition}{Definition}

\title{Expanse Math}

\begin{document}

\maketitle

\section{Artificial gravity}

In science fiction, artificial gravity is often generated via centripital acceleration. Usually (due to production limitations) this ``artificial gravity'' is portrayed as being an excellent facsimile for Earth's gravity. However, some writers have discussed the differences between Earth's gravity and the artifical gravity. In the television series \textit{The Expanse}, the dwarf planet Ceres is given an artificial gravity of $0.3g$ (where $g$ is the acceleration due to gravity on Earth). In one scene, a drink is poured and the liquid appears to follow an interesting path due to the artificial gravity:

%https://blogs-images.forbes.com/chadorzel/files/2016/01/output_wSTCkz.gif
\[
    \includegraphics[width=3in]{pour.png}
\]

Is depiction accurate? To approximate the path the drink will take from the perspective of Miller, we first have to see how Ceres, and therefore Miller, is moving. Afterwards, we can find the path of the drink, and then subtract its position function from Miller's position function to find the drink's path in his point of view. Miller's position function can be modeled by the parametric equation of a circle, \vec{p}, with a radius of $R$ and angular velocity $\omega$.

\[
\vec{p}(t) = \langle R\cos(\omega t),R \sin(\omega t)\rangle
\]

%Ceres' angular velocity can be found using the centripetal acceleration equation, $a = v^2 /r$, and when solved for velocity $v$, is $v = \sqrt{aR}$. From this, we find angular velocity $\omega = v/R = \sqrt{aR}/R$, meaning that on the Ceres, for its artificial gravity $a$ created within it, and its radius $R$, it will have an angular velocity of $\sqrt{aR}/R$, and will complete a full rotation in $\frac{2\pi}{\sqrt{aR}/R}$ seconds.%

To find $\omega$, we will need to take the second derivative of $\vec{p}(t)$, $\vec{p}''(t)$, which will describe Miller's centripetal acceleration.

\[
\vec{p}''(t) = \langle -\omega^2 R \cos(\omega t), - \omega^2 R \sin(\omega t)\rangle
\]

We can then find the magnitude of the acceleration with respect to $\omega$ using the Pythagorean Theorem, $a^2 = x^2 + y^2$. Using the $x$ and $y$ values of $\vec{p}''(t)$ as well as the fact that the acceleration is $0.3g$, we have

\begin{align*}
    a^2 &= x^2 + y^2 \\
    a &= \sqrt{x^2 + y^2} \\
    &= \sqrt{(-\omega^2 R cos(\omega t))^2 + (- \omega^2 R sin(\omega t))^2} \\
    &= \sqrt{\omega^4 R^2 (\sin^2(\omega t) + \cos^2(\omega t))} \\
    &= \sqrt{\omega^4 R^2} \\
    &= \omega^2 R
    \end{align*}
    So we may write
\begin{align*}
    \omega^2 &= a / R\\
    \omega &= \sqrt{a/R}
\end{align*}

Now that we know that $\omega = \sqrt{a/R}$, we can use our knowledge that Ceres will complete one complete rotation in $\frac{2\pi}{\omega}$ seconds and the fact that linear velocity $v$ is related to $\omega$ through $v = \omega R$ to find the position function of the drink.

The drink will be poured a certain distance from the outside of Ceres, so it will have a smaller radius, \(r\). Since this drink is no longer influenced by Ceres' angular velocity after being poured, it will travel tangentially from Ceres' surface at a linear velocity of \(v = \omega r\). Using this information, and assuming the drink is poured at $t = 0$, we have the position function of the drink, $\vec{b}(t)$, as

\[
\vec{b}(t) = \langle r, \sqrt{\frac{a}{R}}rt\rangle
\]

Bringing it together, to find the path the drink takes from the perspective of the person, we have to take the position function of the drink, \vec{b}, and subtract it by the position function of the person, \vec{p}, to get the net distance, \vec{d}(t).

\[
\vec{d}(t) = \vec{b}(t) - \vec{p}(t) = \langle r - R\cos(\sqrt{\frac{a}{R}} t), \sqrt{\frac{a}{R}}rt - R \sin(\sqrt{\frac{a}{R}} t)\rangle
\]

$\vec{n}(t)$ will track the path of the drink until it crosses the y-axis, at which point it would be equivalent to hitting the floor of Ceres. Plugging in the actual values of Ceres, and estimating the drink to be poured $0.25$m above the cup, we have

\[
\vec{d}(t) = 
\begin{bmatrix}
(473000 - 0.25) - 473000\cos(\sqrt{\frac{0.3\cdot9.8}{473000}}t)\\ \sqrt{\frac{0.3\cdot9.8}{473000}}(473000 - 0.25)t - 473000 \sin(\sqrt{\frac{0.3\cdot9.8}{473000}}t)
\end{bmatrix}
\]

Finding when the x-component of $\vec{d}(t) = 0$, we find that $t = \frac{\cos^{-1}(r/R)}{\omega} = 0.41239$ seconds, in the case given above. xSolving for the y-component at $t=0.41239$, we find $y = -1.713 \cdot 10^{-4}$, meaning that casually pouring a drink would only shift the stream less than a fifth of a millimeter from a similar pour on Earth, following a path that resembles the beginning of a cycloid.

These effects can be magnified at greater distances. For example, if something dropped from 10 meters, it would land about 4 centimeters off from its Earth landing position, and from 100 meters, it would be about 1.4 meters off. Whatever the case, the path of any falling object, be it a drink or Miller's hat, does not resemble the path made in The Expanse.

\section{Explosions}

In Epsiode 1, the \textit{Canterbury} is destroyed. 
The \textit{Knight} is then damaged by the wreckage. 
Is this realistic?

The wreckage can be modeled with the vector field:
\[
\vec{F}(x,y,z) = \frac{\vec{x}}{4\pi|\vec{x}|^3}
\]
where $\vec{x} = \langle x,y,z\rangle$

I believe the \textit{Canterbury} is $26000$ from the \textit{Knight}

Based on:

\texttt{https://www.youtube.com/watch?v=qX8TlGCsllM}

It looks like the \textit{Canterbury} is $1000$m long and $250$ m wide. 

The


\section{Cererian Gravity}

Ceres, a dwarf planet in the asteroid belt, has been spun up by Tycho Manufacturing to have a gravitational force of $0.3$g. How fast would Ceres need to spin?

For any large rotating body, its gravity pulls objects toward its center, while the rotation of it counteracts gravity, pushing objects away. If the centrifugal force of the object is greater than its gravity, objects on the surface, and the surface itself, would fly off, tangent to the surface of the planet. To verify whether Ceres can maintain an internal artificial gravity of 0.3g, we can see if Cererian gravity is greater than the artificial gravity or note (Side note: If the artificial gravity was equal to the real gravity, anything on its surface would be weightless, as all of gravity is cancelled out by the centrifugal force). Cererian gravity can be found though the Newton's Law of Universal Gravitation,
\[
g=\frac{GM}{R^2}
\]
where \(G=6.67*10^{-11}\) is the gravitational constant, \(M\) is the mass of the object, and \(R\) is the radius of the object. For Ceres, \(M=9.4*10^{20}\) and \(R=4.73*10^5\), so
\[
g_{Ceres}=\frac{6.67*10^{-11}\cdot 9.4\cdot 10^{20}}{(4.73\cdot10^5)^2}=0.28\frac{m}{s^2}
\]

Since the gravitation of Ceres, $0.28 \frac{m}{s^2}$, is smaller than the centrifugal force, $0.3g$, Ceres wouldn't be able to withstand being spun up to 0.3g without losing a large part of its mass.

Knowing that $a_{centrifugal} = g$ to maintain a stable artificial gravity, many more questions arise. Is there a better candidate to spin up to achieve 0.3g? How big would Ceres look with missing mass, and how strong would the artificial gravity be? We'll take these questions one at a time.

\subsection{A Better Candidate}
In order to spin a planet so its artificial gravity is 0.3g, we need a planet that has a regular gravity over or equal to 0.3g. The closest match? Mars, with \(g_{Mars} = 0.376g\), but with the MCR already claiming the planet, the next best match is Mercury, with a very similar gravity of 0.37g.

\subsection{Ceres-ly Smaller}
Suppose Tycho followed through with spinning up Ceres to the speed necessary for 0.3g, even though it would shed a significant amount of mass. How big would this new Ceres look, and what would the artificial gravity be like inside of it?

First, we need to find the velocity of Ceres at its equator, and by manipulating \(a = \frac{v^2}{R}\), we get \(v = \sqrt{aR}= \sqrt{0.3g * 4.73 * 10^5} = 1179\frac{m}{s}\). We also know that \(\omega = \frac{v}{R} = \frac{1179}{4.73*10^5} = 0.00249\frac{rad}{s}\). Using \(v=\omega  r\) and \(a_{centripetal} = g = \frac{v^2}{r} = \frac{GM}{r^2}\), we now have 
\[
\frac{(\omega r)^2}{r} = \frac{G M}{r^2}\quad
\Leftrightarrow \quad r = \sqrt[3]{\frac{GM}{\omega}},
\]
giving us the radius of the new Ceres, which is 
\[
\sqrt[3]{
\frac{6.67*10^{-11}*9.4*10^{20}}{0.00249^2}
} = 29316m.
\]
For perspective, that's similar to something the size of a car tire disintegrating into something the size of a quarter.

Next, we need to find the mass of the smaller Ceres. We'll assume that Ceres is round and has uniform density. Given that density \(\rho = \frac{M_{old}}{V_{old}}\), where \(V=\frac{4}{3}\pi r^3\) is the volume of a sphere, we can find the new mass as 
\begin{align*}
    M_{new} &= \rho V_{new}\\
    &= \frac{M_{old}}{\frac{4}{3}\pi r_{old}^3}\cdot \frac{4\pi r_{new}^3}{3} \\
    &= \frac{M_{old}r_{new}^3}{r_{old}^3} \\
    &= \frac{9.4\cdot10^{20} \cdot 29316^3}{(4.73\cdot 10^5)^3} \\
    &= 2.24\cdot10^{17}kg.
\end{align*}

Finally, to find the maximum artificial gravity of the smaller Ceres, we just need to find the surface gravity of it, which would be \(\frac{GM_{new}}{r_{new}^2}=\frac{6.67*10^{-11}*2.24*10^{17}}{(29316)^2} = 0.017\frac{m}{s^2} = 0.0018g\), falling short of the 0.3g goal by quite a large amount.

\section{More Questions \& Thoughts}
\begin{enumerate}
    \item How do the magnetic boots work? Can you "turn off" magnetism in boots?
    \subitem How would it feel? Additional manipulation of the magnetic field?
    
    \item Season 1, Episode 7. Are the vertical windmills in the back of Holden's house more efficient than traditional wind turbines?
    
    \item Shed's blood in 0g, would it actually congregate into a ball?
    \item How quickly would air evacuate with a hole in the wall?
    \subitem How long until Nauvoo lost all air?
    \item How strong was the radiation that hit Miller and Holden, based on symptoms and time exposed?
    \item Realistic travel times? From Tachi/Rocinante to Tycho, Tycho to Anubis to Eros, Phoebe to Eros? Earth to Phoebe?
    \subitem How much fuel would be needed?
    \item Spread of disease in the second season?
    \item How much power would it take to blow up Phoebe/Deimos?
    \item Belter bones + falling from switch of 0g to 0.3g. Anything broken?
    \item Nikil fell down out of the airlock. Shouldn't he have fallen sideways?
    \item What would be needed for Martians to train at 1g? On a spaceship? Wouldn't that crush Martian bones like Belters in 1g?
    \subitem Possibly a ship travelling in circles? how big is that circle?
    \item What's the circumference of the Belt?
    \item How many lightseconds is the Belt from Earth/Mars, at closest and farthest distances? (For communication delay purposes)
    \item How long would the Nauvoo trip take? Where are they going?
    \item What's the max velocity of a trip of distance x? (given it's 0.3g acceleration)
    \item The Nauvoo-Eros collision course for the Sun is inaccurate.
    \subitem Given that it's accurate, how fast would the Nauvoo be when it would it Eros?
    \subitem how long would it take to plummet into the Sun?
    \item How strong are the suit thrusters? How many are there? Waste of matter?
    \item How heavy can something be in 0g? (Like Miller's bomb?)
    \item How fast was the Marasmus debris hitting Miller?
    \item How much force needed to more Eros 1 radius?
    \subitem What is the radius of Eros?
    \item What's the population of Eros?
    \subitem Is the disease spread time realistic? How long should it take?
    \item How fast is Eros moving?
    \item Power of Miller's nuke?
    \item Are face lights in a suit actually effective, or just cinematic necessities?
    \item Would it be hard to cool electronics in a vacuum?
    \item Energy required to get out of the Hill radius?
    \item Fuel needed to travel between places? time? min/max?
    \item Time needed to travel from Mars to Phoebe? from Earth? What positioning would be needed to get there at the same time?
    \item Ceres population? How much water needed to keep it alive?
    \item How much water/air/acres of land for food needed for one human for one day?
    \item How much water could the Canterbury hold? How big is the Cant/Roci?
    \item Are trees on Ceres efficient? (use water, give oxygen)
    \item Escape velocity of moon? Mars? Belt asteroids?
    \item 13 minute delay in ep 5, how far away is Roci and Eros from Earth?
    \item Would the Nauvoo take that long to zoom by Eros?
    \item Could you throw a punch in 0.3g? 0g?
    \item How much more dense would Mars need to be to hold an Earth-like atmosphere?
    \item Given density of matter in space, how many holes should the Nauvoo have by the end of its trip? Is that realistic for survival?
    \item Surface area of Nauvoo?
    \item Martian population?
    \item Force needed to break a rock?
    \item g's needed to kill a human? Knock them out?
    \item How fast was Bizi Betiko going to get killed?
    
    \item Do scoped earth guns work on mars?scopes have equally spaced ridge lines that allow the shooter to accpunt for a bullet's drop. Would these scoped guns work on mars, where the gravity is less than that of earth's?
    \item Continuation of question above...how would a bullet behave on a ship with artificial gravity (generated by centrifugal force). How would you build a scope for a gun in such a case?
    \item Turn on a dime in space? In The Expance, small crafts rotate with ease. How do real space craft manage to rotate?
    \item How do you brew coffee in zero G?
    \item what is the accelertion of the ship at 28:48? assume the camera is not acceleratin and that [part to be determined] is approximately [size to be determined].
    \item Crew members use magnetic shoes in zero G environments. What is the force on their ankles? How would that force change if they used knee length rigid magnetic boots instead?
    \item In episode 1 at 40:46, the low gravity environment is made apparent by a bird that seems to float. can you calculate the acceleration of the bird as it drops? Are there any split second counter examples of objects falling, such that their velocity is incomsistent with the bird falling?
    
\end{enumerate}
if the\end{document}
